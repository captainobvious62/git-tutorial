\RequirePackage[l2tabu, orthodox]{nag}
\documentclass[12pt]{article}

\usepackage{amssymb,amsmath,verbatim,graphicx,microtype,upquote,units,booktabs,siunitx,xcolor,fancyvrb,newverbs}
\usepackage[utf8]{inputenc}
\usepackage[margin=10pt, font=small, labelfont=bf, labelsep=endash]{caption}
\usepackage[
            colorlinks = true,
            linkcolor = blue,
            urlcolor  = blue,
            citecolor = blue,
            anchorcolor = blue]{hyperref}


\definecolor{cverbbg}{gray}{.7}
\newenvironment{lcverbatim}
 {\SaveVerbatim{cverb}}
 {\endSaveVerbatim
  \flushleft\fboxrule=0pt\fboxsep=.5em
  \scriptsize
  \colorbox{cverbbg}{%
    \makebox[\dimexpr\linewidth-2\fboxsep][l]{\BUseVerbatim{cverb}}%
  }
  \endflushleft
}


\newcommand{\shellcmd}[1]{\texttt{\colorbox{gray!30}{#1}}}

\title{Fundamentals of Git}
\date{\today}
\author{Illya Starikov}

\begin{document}
\maketitle


\section{\texorpdfstring{\shellcmd{Git}ting} \ \ Good}
Let's start with a blank directory\footnote{We don't always have to do this, especially if we already have a git repository somewhere else.}. After opening your terminal emulator of choice,

\begin{enumerate}
    \item Navigate to your desktop directory via \shellcmd{cd}.
    \item Make a new directory named \texttt{git-tac-toe} for your git repository (\shellcmd{mkdir git-tac-toe}) and change into said directory (\shellcmd{cd git-tac-toe}).
    \item Create a git repository within this directory (\shellcmd{git init}).
\end{enumerate}

\noindent We are now ready to start using Git! \texttt{Initialized empty Git repository in $\cdots$} or something of that nature should appear. If you receive a warning/error message, changes are git was not set up correctly.

A common convention is to have a \texttt{README.md} file in your directory. We will create one now.

\begin{enumerate}
    \item Create a new file named \texttt{README.md} with your text editor of choice (for this demo, I will be using \href{http://www.vim.org}{vim}).
    \item Insert the following text into the file --- if you're not familiar with Markdown, the syntax is easy. Check it out \href{https://guides.github.com/features/mastering-markdown/}{here}.
\end{enumerate}

\begin{lcverbatim}
# Git Tac Toe

This is just a demonstration on the fundamentals of git!
\end{lcverbatim}

\begin{enumerate}
    \setcounter{enumi}{2}
    \item Save your changes. At this point, we have enough to commit our changes. Add the markdown file with \shellcmd{git add README.md}.
    \item Commit your files with \shellcmd{git commit}. A text editor should appear asking you for your commit message (if you've never used Git nor done anything with your environmental variables, it will probably be nano).
    \begin{itemize}
        \item If you want to change the Git text editor, \\ \shellcmd{git config --global core.editor "EDITOR"}.
        \item If you want to change the global text editor, \shellcmd{export VISUAL=vim} followed by \shellcmd{export EDITOR="\$VISUAL"}.
    \end{itemize}

    \item You commit message should appear like so:
\end{enumerate}

\begin{lcverbatim}
Added a read me to describe project
# Please enter the commit message for your changes. Lines starting
# with '#' will be ignored, and an empty message aborts the commit.
# On branch master
#
# Initial commit
#
# Changes to be committed:
#       new file:   README.md
\end{lcverbatim}

\noindent Note the commit message isn't particularly long; that's because the intent is very clear.

\section{Branching And Merging}


\end{document}
