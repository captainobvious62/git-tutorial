\RequirePackage[l2tabu, orthodox]{nag}
\documentclass[12pt]{article}

\usepackage{amssymb,amsmath,verbatim,graphicx,microtype,upquote,units,booktabs,siunitx,xcolor,fancyvrb,newverbs,textcomp}
\usepackage[utf8]{inputenc}
\usepackage[margin=10pt, font=small, labelfont=bf, labelsep=endash]{caption}
\usepackage[
            colorlinks = true,
            linkcolor = blue,
            urlcolor  = blue,
            citecolor = blue,
            anchorcolor = blue]{hyperref}


\definecolor{cverbbg}{gray}{.7}
\newenvironment{lcverbatim}
 {\SaveVerbatim{cverb}}
 {\endSaveVerbatim
  \flushleft\fboxrule=0pt\fboxsep=.5em
  \scriptsize
  \colorbox{cverbbg}{%
    \makebox[\dimexpr\linewidth-2\fboxsep][l]{\BUseVerbatim{cverb}}%
  }
  \endflushleft
}


\newcommand{\shellcmd}[1]{\texttt{\colorbox{gray!30}{#1}}}

\title{Fundamentals of Git}
\date{\today}
\author{Fearless Leader}

\begin{document}
\maketitle

These are the show notes of how to use Git for the Aminzadeh Research Group. This accompany the lecture slides, and each section corresponds to where the words ``Demo'' appear.

There is some maintenance that we have to take care of before we can begin. It is as follows.

\begin{enumerate}
    \item Copy the directory \texttt{git-tac-toe}\footnote{\texttt{demo} \textrightarrow \ \texttt{tic-tac-toe}} to a temporary location (Desktop should work fine).
    \item Navigate to said directory in you terminal emulator of choice.
    \item Run the commands \\ \shellcmd{git reset --hard 79686d3ae905dc01b1015bd9387dcbadbf6826da} followed by \shellcmd{git branch -D bug-fixes}. 
\end{enumerate}

\section{\texorpdfstring{\shellcmd{Git}ting} \ \ Good}
Let's start with a blank directory\footnote{We don't always have to do this, especially if we already have a git repository somewhere else.}. After opening your terminal emulator of choice,

\begin{enumerate}
    \item Navigate to your desktop directory via \shellcmd{cd}.
    \item Make a new directory named \texttt{git-tac-toe} for your git repository (\shellcmd{mkdir git-tac-toe}) and change into said directory (\shellcmd{cd git-tac-toe}).
    \item Create a git repository within this directory (\shellcmd{git init}).
\end{enumerate}

\noindent We are now ready to start using Git! \texttt{Initialized empty Git repository in $\cdots$} or something of that nature should appear. If you receive a warning/error message, changes are git was not set up correctly.

A common convention is to have a \texttt{README.md} file in your directory. We will create one now.

\begin{enumerate}
    \item Create a new file named \texttt{README.md} with your text editor of choice (for this demo, I will be using \href{http://www.vim.org}{vim}).
    \item Insert the following text into the file --- if you're not familiar with Markdown, the syntax is easy. Check it out \href{https://guides.github.com/features/mastering-markdown/}{here}.
\end{enumerate}

\begin{lcverbatim}
# Git Tac Toe

This is just a demonstration on the fundamentals of git!
\end{lcverbatim}

\begin{enumerate}
    \setcounter{enumi}{2}
    \item Save your changes. At this point, we have enough to commit our changes. Add the markdown file with \shellcmd{git add README.md}.
    \item Commit your files with \shellcmd{git commit}. A text editor should appear asking you for your commit message (if you've never used Git nor done anything with your environmental variables, it will probably be nano).
    \begin{itemize}
        \item If you want to change the Git text editor, \\ \shellcmd{git config --global core.editor "EDITOR"}.
        \item If you want to change the global text editor, \shellcmd{export VISUAL=vim} followed by \shellcmd{export EDITOR="\$VISUAL"}.
    \end{itemize}

    \item You commit message should appear like so:
\end{enumerate}

\begin{lcverbatim}
Added a read me to describe project
# Please enter the commit message for your changes. Lines starting
# with '#' will be ignored, and an empty message aborts the commit.
# On branch master
#
# Initial commit
#
# Changes to be committed:
#       new file:   README.md
\end{lcverbatim}

\noindent Note the commit message isn't particularly long; that's because the intent is very clear.

\section{Branching And Merging}
Now we'll discuss the branching and merging. Please be sure to complete the instructions at the top of this document.

Let's create our game, Git-Tac-Toe.

\begin{enumerate}
    \item Switch to the branch \texttt{game} via \shellcmd{git checkout game}. As you can see, it's already finished! All we have to do is merge!
    \item Re-checkout the master branch via \shellcmd{git checkout master}. Merge with \shellcmd{git merge game}.
    \item Uh-oh! Merge conflict. Reading the message, we notice it's in \texttt{README.md}. We can fix it by deleting one of the two lines; either one is fine.
\end{enumerate}

\begin{lcverbatim}
CONFLICT (content): Merge conflict in README.md
Automatic merge failed; fix conflicts and then commit the result.
\end{lcverbatim}

\begin{enumerate}
    \setcounter{enumi}{3}
    \item Re-add \texttt{README.md} and commit. Notice the fancy commit message.
\end{enumerate}

\begin{lcverbatim}
Merge branch 'game'

# Conflicts:
#       README.md
#
# It looks like you may be committing a merge.
# If this is not correct, please remove the file
#       .git/MERGE_HEAD
# and try again.


# Please enter the commit message for your changes. Lines starting
# with '#' will be ignored, and an empty message aborts the commit.
# On branch master
# All conflicts fixed but you are still merging.
#
# Changes to be committed:
#       new file:   .gitignore
#       modified:   README.md
#       new file:   board.cpp
#       new file:   board.hpp
#       new file:   functions.cpp
#       new file:   functions.hpp
#       new file:   main.cpp
\end{lcverbatim}

\begin{enumerate}
    \setcounter{enumi}{4}
    \item Compile and run the code (\shellcmd{g++ *.cpp -std=c++14}). Notice after a move, it breaks. Time for debugging.
    \item Create a new branch (\shellcmd{git branch bug-fixes}) and checkout said branch (\shellcmd{git checkout bug-fixes}).
    \item Open \texttt{functions.hpp} and search for ``error''; two results should appear.
\end{enumerate}

\begin{lcverbatim}
    for (const auto& subarray: toFlatten) {
        // This will definitely cause an error
        // result.insert(result.end(), subarray.begin(), subarray.end());
    }
\end{lcverbatim}

\begin{lcverbatim}
    for (int i = 0; i < std::max(vectorT.size(), vectorS.size()); i++) {
        // subtract 1 because size is not 0-indexed
        
        /* This will also definitely cause an error
        if (vectorS.size() - 1 < i) {
            result.push_back(std::make_tuple(vectorT.at(i), defaultS));
        } else if (vectorT.size() - 1 < i) {
            result.push_back(std::make_tuple(defaultT, vectorS.at(i)));
        } else {*/
            result.push_back(std::make_tuple(vectorT.at(i), vectorS.at(i)));
        //}
    }
\end{lcverbatim}

\begin{enumerate}
    \setcounter{enumi}{7}
    \item Fix these errors one a time, committing in between. Recompile to verify that it works.
    \item Checkout the master branch (\shellcmd{git checkout master}) and merge \\ (\shellcmd{git merge bug-fixes}).
\end{enumerate}

\end{document}
