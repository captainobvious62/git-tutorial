\documentclass{beamer}

\usepackage{amssymb,amsmath,verbatim,graphicx,microtype,units,booktabs,upquote,xcolor,siunitx,csquotes,fancyvrb,newverbs}
\hypersetup{
            colorlinks = true,
            linkcolor = blue,
            urlcolor  = blue,
            citecolor = blue,
            anchorcolor = blue
}

\title{Fundamentals of Git}
\subtitle{Missouri Satellite Team}
\author{Presented by: Illya Starikov}
\date{ }

\newcommand{\shellcmd}[1]{\texttt{\colorbox{gray!30}{#1}}}

\definecolor{cverbbg}{gray}{.7}
\newenvironment{lcverbatim}
 {\SaveVerbatim{cverb}}
 {\endSaveVerbatim
  \flushleft\fboxrule=0pt\fboxsep=.5em
  \colorbox{cverbbg}{%
    \makebox[\dimexpr\linewidth-2\fboxsep][l]{\BUseVerbatim{cverb}}%
  }
  \endflushleft
}

\begin{document}
\begin{frame}
    \maketitle
\end{frame}

\begin{frame}
    \frametitle{Getting Started}

    There'll be some setting up before you can start using git. It'll depend on your operating system --- you can refer to the installation guide
    \href{https://git-scm.com/book/en/v2/Getting-Started-Installing-Git}{here}.
    Below are the more popular methods of installation.

    \begin{description}
        \item[macOS] \shellcmd{brew install git}\footnote{Or whatever hip package manager you use.}
        \item[Linux] Depends on your distro. If Ubuntu use \shellcmd{sudo apt-get install git-all}, if Arch Linux then \shellcmd{pacman -S git}, if others refer \href{https://git-scm.com/download/linux}{here}.
        \item[Windows] Download a \texttt{.exe} from \href{https://git-scm.com/download/win}{here}.
    \end{description}

\end{frame}

\end{document}
