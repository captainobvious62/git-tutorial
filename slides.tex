\documentclass{beamer}

\usepackage{amssymb,amsmath,verbatim,graphicx,microtype,units,booktabs,upquote,xcolor,siunitx,csquotes,fancyvrb,newverbs,wrapfig,multicol}
\hypersetup{
            colorlinks = true,
            linkcolor = blue,
            urlcolor  = blue,
            citecolor = blue,
            anchorcolor = blue
}

\title{Fundamentals of Git}
\subtitle{Missouri Satellite Team}
\author{Presented by: Illya Starikov}
\date{ }

\newcommand{\shellcmd}[1]{\texttt{\colorbox{gray!30}{#1}}}

\definecolor{cverbbg}{gray}{.7}
\newenvironment{lcverbatim}
 {\SaveVerbatim{cverb}}
 {\endSaveVerbatim
  \flushleft\fboxrule=0pt\fboxsep=.5em
  \colorbox{cverbbg}{%
    \makebox[\dimexpr\linewidth-2\fboxsep][l]{\BUseVerbatim{cverb}}%
  }
  \endflushleft
}

\newcommand{\hugeslide}[1]{
\begin{frame}[plain,c]
    \centering {\usebeamerfont*{frametitle} \usebeamercolor[fg]{frametitle}\fontsize{40}{50}\selectfont #1}
\end{frame}
}
\begin{document}
\begin{frame}
    \maketitle
\end{frame}

\begin{frame}
    \frametitle{Getting Started}

    There'll be some setting up before you can start using git. It'll depend on your operating system --- you can refer to the installation guide
    \href{https://git-scm.com/book/en/v2/Getting-Started-Installing-Git}{here}.
    Below are the more popular methods of installation.

    \begin{description}
        \item[macOS] \shellcmd{brew install git}\footnote{Or whatever hip package manager you use.}
        \item[Linux] Depends on your distro. If Ubuntu use \shellcmd{sudo apt-get install git-all}, if Arch Linux then \shellcmd{pacman -S git}, if others refer \href{https://git-scm.com/download/linux}{here}.
        \item[Windows] Download a \texttt{.exe} from \href{https://git-scm.com/download/win}{here}.
    \end{description}

\end{frame}

\begin{frame}
    \frametitle{What is Git?}

    \begin{multicols}{2}

        \begin{itemize}
            \item Git is nothing more than Directed Acyclic Graph of objects compressed and identified by an SHA-1 hash. % What?
            \item Git works in snapshots, not differences. % Meaning when you commit (or save), you save a mini filesystem at the time of commit. This makes going through and seeing changes very easy
                \item Git is local. % Changes can be saved and continue your work no matter where you are. You can't receive/transmit changes from the remote, central repository, but you can continue your work (when you are ready to pull changes from central repo, it'll merge).
            \item Git has data integrity. % That's the SHA-1 hash. All your snapshots are referred to by a checksum.
            \item Git is parallelizable. % Meaning hundreds of people can work on hundreds different versions of the repository, and it'll still work.
        \end{itemize}

        \begin{center}
            \includegraphics[width=0.35\textwidth]{xkcd}
        \end{center}
    \end{multicols}
\end{frame}

\begin{frame}
    \frametitle{The Five Stages of Git}

    \begin{enumerate}
        \item Working Directory % This is the local repository. The changes and modification you make are in this stage, until you 'add' (i.e. specifically say these are the files whos changes you want to `commit') them.
        \item Staging Area % When files are 'added', they are in the staging area, you ready for you to say `these are the changes I made'.
        \item Git Directory % After `committing' (nothing more than saying I made these changes, here's what's different), they are updated in the git databases
        \item $\cdots$
        \item Profit % fuck yeah!
    \end{enumerate}
\end{frame}

\begin{frame}
    \frametitle{\shellcmd{Git}ting Good}

    If the ``stages'' didn't make sense, that's alright. It's better to go through a workflow as apposed to the formalities.

    \begin{enumerate}
        \item Start a new repository with \shellcmd{git init} % this creates your working directory
        \item Work on project in bite sized chunks, and add files that were changed with \shellcmd{git add file(s)} % adding to the staging area
        \item Commit your changes with \shellcmd{git commit} % This transfers them to the Git directory
        \item Optionally, \shellcmd{git push} to save changes to the remote branch
        \item Of course, profit.
    \end{enumerate}
\end{frame}

\hugeslide{Demo}



\end{document}
